\documentclass[12pt]{article}

\usepackage{sbc-template}

\usepackage{graphicx,url}

\usepackage[brazil]{babel}   
\usepackage[latin1]{inputenc}  

     
\sloppy

\title{Instructions for Authors of SBC Conferences\\ Papers and Abstracts}

\author{Luciana P. Nedel\inst{1}, Rafael H. Bordini\inst{2}, Fl�vio Rech
  Wagner\inst{1}, Jomi F. H�bner\inst{3} }


\address{Instituto de Inform�tica -- Universidade Federal do Rio Grande do Sul
  (UFRGS)\\
  Caixa Postal 15.064 -- 91.501-970 -- Porto Alegre -- RS -- Brazil
\nextinstitute
  Department of Computer Science -- University of Durham\\
  Durham, U.K.
\nextinstitute
  Departamento de Sistemas e Computa��o\\
  Universidade Regional de Blumenal (FURB) -- Blumenau, SC -- Brazil
  \email{\{nedel,flavio\}@inf.ufrgs.br, R.Bordini@durham.ac.uk,
  jomi@inf.furb.br}
}

\begin{document} 

\maketitle

\begin{abstract}
  Do we need an abstract here?
\end{abstract}
     
\begin{resumo} 
  Precisamos de um resumo aqui?
\end{resumo}


\section{Contextualiza��o e motiva��o do tema}

Aqui diria por que Data Warehouses s�o legais

\section{Principais conceitos relacionados ao tema} \label{sec:firstpage}

Dar uma explicada com base no livro de Data Warehouse o que s�o elas e

\section{Descri��o do estado da arte de trabalhos te�ricos e tamb�m de ferramentas de software relacionadas ao tema}

Falar as ferramentas que existem, mostrar o SpagoBI, o Vertica e talvez

algum outro se achar artigos

\section{Um pouco mais de detalhes de ao menos uma ferramenta escolhida para ilustrar a implementa��o de conceitos abordados no trabalho}

Aqui come�amos a mostrar por que o Vertica � legal. Armazenamento


\section{Um estudo de caso, reportando o uso da ferramenta escolhida para a cria��o e manipula��o de um banco de dados de exemplo.}\label{sec:figs}

Rodar o Vertica e escrever (com imagens) como se usa, desde os dados no banco relacional at� a consulta\cite{smith:99}.

%\begin{figure}[ht]
%\centering
%\includegraphics[width=.5\textwidth]{fig1.jpg}
%\caption{A typical figure}
%\label{fig:exampleFig1}
%\end{figure}

%\begin{figure}[ht]
%\centering
%\includegraphics[width=.3\textwidth]{fig2.jpg}
%\caption{This figure is an example of a figure caption taking more than one
  %line and justified considering margins mentioned in Section~\ref{sec:figs}.}
%\label{fig:exampleFig2}
%\end{figure}

%\begin{table}[ht]
%\centering
%\caption{Variables to be considered on the evaluation of interaction
  %techniques}
%\label{tab:exTable1}
%\includegraphics[width=.7\textwidth]{table.jpg}
%\end{table}


\bibliographystyle{sbc}
\bibliography{datawarehouse}

\end{document}
